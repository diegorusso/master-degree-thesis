\begin{abstract}
In the last decade \textit{Deep Learning} has had a incredible success due to
the increasing availability of big data and the decreasing cost of
computational power.
A massive amount of  research effort has regarded the investigation of new
neural network models for application to different domains, and an increasing
interested has arisen techniques on for optimizing deployment of such models to
edge devices with limited resources.
The aim of these techniques is to optimize the generated deep learning models
to be more efficient with respect to computational power and memory
requirements while having no or little loss in accuracy, thus allowing
efficient deployment of a neural network model to edge devices.
The aim of this thesis is to investigate and validate a specific pruning
technique: layers of a network model are pruned based on a heuristic whilst
respecting the target sparsity of the network model.
After introducing the theory behind this approach, the application design is
presented and its implementation experimented to a simple MNIST based network
model using TensorFlow Model Optimization.
Experiments have been held on MobileNet v1 architecture using CIFAR-10 and
ImageNet 2012 as datasets.
Experimental results of the pruning technique are presented, in particular we
observe that, the heuristic distribution of weights behave more robustly
compared to the uniform distribution especially with higher sparsity levels.
\end{abstract}