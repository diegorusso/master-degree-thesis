\chapter{Conclusions}
Pruning is an excellent technique of model optimization: it allows to
dramatically decrease the size of a model with no or low impact in accuracy.

In this thesis I have demonstrated the benefits of the heuristic distribution
of weights while pruning a model: layers with mode weights can be pruned more
compared to layers with less weights while maintaining the final sparsity
specified by the user.

To further optimise the model, quantization can be applied: the non-uniform
distribution of the weights though doesn't affect the quantization process.

The heuristic distribution results more robust compared to the uniform
distribution of weights: in MobileNet v1, for both CIFAR-10 and ImageNet the
heuristic distribution of weights always outperforms the uniform distribution.
This is true from a sparsity level of 0.5 up to 0.9: especially in the CIFAR-10
case, this difference increases with higher sparsity levels giving a difference
in accuracy of 2.14\%.

The work in this thesis represents a solid base for enabling non-uniform
distributions of weights in TensorFlow Model Optimization.

As next steps, I have identified the following:

\begin{itemize}
    \item \textbf{Upstream code to GitHub}: using this thesis as starting
        point, create an RFC proposal and engage with Google to receive
        feedback. Once details have been agreed, a pull request in GitHub
        can be raised in order to be merged in TFMOT\@.
        As part of the pull request, user documentation, testing (unit tests
        and end-to-end tests) need to be written.
    \item \textbf{Provide more off-the-shelf heuristics}: the architecture of
        the code has been designed in a way to be as generic as possible: the
        user can implement his/her own distributions of weights and pass
        it to the pruning parameters.
        As extra step, more off-the-shelf heuristic distributions can be
        offered to the user: Erd\H{o}s-R\'{e}nyi and Erd\H{o}s-R\'{e}nyi-Kernel
        (ERK)\cite{rigl}
    \item \textbf{Expand tests on new architecture/datasets}: this thesis
        focussed only on MobileNet v1 architecture with CIFAR-10 and ImageNet
        datasets. In order to further prove the benefits of the heuristic
        distribution of weights, tests on more comprehensive set of
        architectures and datasets need to be performed.
\end{itemize}
