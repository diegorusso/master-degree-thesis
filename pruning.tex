\chapter{Pruning}\label{ch:pruning}
In \autoref{ch:introduction} I gave a brief explanation of few techniques for
doing model optimisation on a neural network. Pruning is one of these and in
the first part of this chapter I give a more detailed explanation.
The second part instead focuses on the core of the thesis: per-layer pruning
configuration with heuristic.

Despite being technical, this chapter is still fairly theoretical and I defer
any code and implementation to \autoref{ch:implementation}.

\section{What's Pruning?}
Neural network pruning is the task of reducing the size of a network by
removing parameters. This compression affects the size of the model, the
latency, the amount of memory and the compute power needed to run the
inference. These metrics need to balanced with the accuracy of the model
itself. I give a more detailed analysis about this trade-off in
\autoref{subsec:tradeoff}

Pruning has been used since the late 1980s but has seen an explosion of
interest in the past decade thanks to the rise of deep neural networks.
It sets its roots with a couple of classic papers:\textit{Optimal Brain
Damage}~\cite{lecun-90b} and \textit{Optimal Brain Surgeon}\cite{hassibi-93}

In the last decade (2010\-2020) a few dozens of papers have been published in
literature about pruning and all of them have been showing that pruning is an
effective technique that can be applied to a variety of neural network on
different fields (image and speech recognition, text processing, etc\ldots).
Moreover they highlights that pruning is a versatile technique as, I said
earlier, it has a positive impact on multiple metrics, all important for a
better edge deployment of the model.

How does pruning reduce the size of a model? The basic principle is to prune
(remove) unnecessary neurones or weights (see \autoref{fig:pruning_weights_neurons}):

\begin{figure}[ht]
    \includegraphics[width=\textwidth]{images/pruning/pruning_weights_neurons.png}
    \centering
    \caption{Pruning weights and neurons}\label{fig:pruning_weights_neurons}
\end{figure}

\begin{itemize}
    \item \textbf{weights}: this is done by setting individual parameters to
        zero and making the network sparse. The effect will be to maintain the
        same architecture of the network but lowering down the number of
        parameters.
    \item \textbf{neurons}: this is done by removing the entire node from the
        network with all its connections. This would make the network
        architecture smaller but with the target to keep the accuracy of the
        starting network.
\end{itemize}

\subsection{Pruning techniques}
The main problem in pruning is to understand what to prune. Of course the goal
is to remove nodes and/or weights that are less useful. There are different
methods to understand what to prune with very little or no effect on accuracy.
Below e brief description of different techniques.

\subsubsection{Magnitude Pruning}
Functions can be a very simple case of a neural network. Its coefficients can
be changed in order to learn the input data points. There are coefficients
that, despite changing their values, they won't change the behaviour of the
function and these are referred as \textbf{non-significant}.
In neural networks these coefficients are weights and biases: they are
\textbf{trainable parameters} and the same non-significant concept can be
applied to them with bit more complexity.

During the back-propagation (gradient descent) some weights are updated with
larger gradient magnitudes (both positive and negative) than the others.
These weights are the \textbf{significant} ones and the weights receiving very
small gradients can be considered as \textbf{non-significant} as their impact
is minimal to the optimization of the loss function.
After the training, the weight magnitude of every layer can be explored and
check which weights are significant.

So the weight magnitude is the criteria for pruning the neural network.
At this point a \textbf{threshold} is specified and all the weights below this
threshold are considered non-significant. This is usually combined with a
\textbf{sparsity target} the network should achieve.
The \textbf{non-significant weights will be zeroed}, cancelling effectively
their impact in the neural network.
This can be applied to biases as well (to any trainable parameter to be
precise).

Once the pruning is done it's always advisable to retrain the network in order
to compensate for any drop in performance. It's worth noticing that during the
retraining the pruned weights won't be updated.~\cite{magnitude_pruning}

Magnitude pruning is the technique I will be using in
\autoref{ch:implementation}.

\subsubsection{Channel Pruning}
\lipsum[1]

\subsubsection{Structured Pruning}
\lipsum[1]

\subsection{Pruning pipeline}
Training, pruning, fine-tuning
1506.02626 \- Learning both Weights and Connections for Efficient Neural Networks
\lipsum[1]

\subsection{Practical examples with Pruning}
\lipsum[1]

\subsection{Sparsity-accuracy trade-offs}\label{subsec:tradeoff}
\lipsum[1]

\section{Per-Layer Pruning Configuration With Heuristic}\label{sec:heuristic}
\lipsum[1]

\subsection{What's a heuristic?}
\lipsum[1]

\subsection{Heuristic details}
Explain in details the technique and the reasons behind
\lipsum[1]

\subsection{Custom heuristic formula}
\lipsum[1]
